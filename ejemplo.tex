\documentclass[a4paper]{article} % Formato de plantilla que vamos a utilizar

\usepackage[utf8]{inputenc}
\usepackage[spanish]{babel}
\usepackage[margin=2cm, top=2cm, includefoot]{geometry}
\usepackage{graphicx} % Para insertar imagenes 
\usepackage[table,xcdraw]{xcolor} % Para la deteccion de colores
\usepackage[most]{tcolorbox} % Para la insercion de cuadros en la portada
\usepackage{fancyhdr} % Para los estilos
\usepackage[hidelinks]{hyperref} % Gestion de hypervinculos
\usepackage{listings} % insertar codigo en el documento
\usepackage{parskip} % Arreglo de la sangria en el documento
\usepackage[figurename=Figura]{caption} % Cambiar el nombre del caption de las fotos
\usepackage{smartdiagram} % Insertar Diagramas
\usepackage{zed-csp} % Insercion de tablas

\usepackage{fontspec} 
\setmainfont{Arial}


% Declaracion de colores
\definecolor{greenPortada}{HTML}{69A84F}

% Declaracion de variables
\newcommand{\logo}{~/null/2021-11-07-213216_635x221_scrot.png}
\newcommand{\machineName}{no tengo nada que hacer}
\newcommand{\logoMachine}{~/null/pexels-tima-miroshnichenko-5380664.jpg}
\newcommand{\startDate}{08 de noviembre del 2021}

% Adicionales
% \addto\captionsspanish{\renewcommand{\contentsname}{Indice}} % Cambio del formato a ESP
\setlength{\headheight}{20.2pt} % size de la barrita que aparece
% \setlength{\headheight}{40.2pt}
\pagestyle{fancy}
\fancyhf{}

% \lhead{\includegraphics[width=4cm]{\logo}} % para poner el logo en las cabezera
\lhead{\normalfont { }}
% \lhead{\rmfamily { }}
% \lhead{\sffamily { }}
% \lhead{\tffamily { }}
% \lhead{\mdseries { }}
% \lhead{\bfseries{ }}
% \lhead{\upshape { }}
% \lhead{\itshape { }}
% \lhead{\scshape { }}
\rhead{\slshape {Informatica Administrativa}}
% \rhead{\includegraphics[height=0.5cm]{\logoMachine}}
\renewcommand{\headrulewidth}{1pt} % Redefinimos la anchura de la barra superior
\renewcommand{\headrule}{\hbox to\headwidth{\color{greenPortada}\leaders\hrule height \headrulewidth\hfill}}
\renewcommand{\lstlistingname}{Codigo} % Para cambiar el caption de los codigos

% Insertar codigo
\definecolor{codegreen}{rgb}{0,0.6,0}
\definecolor{codegray}{rgb}{0.5,0.5,0.5}
\definecolor{codepurple}{rgb}{0.58,0,0.82}
\definecolor{backcolour}{rgb}{0.95,0.95,0.92}

\lstdefinestyle{mystyle}{
    backgroundcolor=\color{backcolour},   
    commentstyle=\color{codegreen},
    keywordstyle=\color{magenta},
    numberstyle=\tiny\color{codegray},
    stringstyle=\color{codepurple},
    basicstyle=\ttfamily\footnotesize,
    breakatwhitespace=false,         
    breaklines=true,                 
    captionpos=b,                    
    keepspaces=true,                 
    numbers=left,                    
    numbersep=5pt,                  
    showspaces=false,                
    showstringspaces=false,
    showtabs=false,                  
    tabsize=2
}

\lstset{style=mystyle}

% Inicializamos el documento
\begin{document}
    % --------------------------------------------------------
    \cfoot{\thepage} % Numeros de pagina
    % --------------------------------------------------------
    % Portada
    \begin{titlepage}
    \centering

    % --------------------------------------------------------
    \includegraphics[width=0.9\textwidth]{\logo}\par\vspace{1cm}
    {\scshape\LARGE \textbf{Informe}\par}
    % --------------------------------------------------------
    \vspace{0.3cm}
    % --------------------------------------------------------
    {\Huge\bfseries\textcolor{greenPortada}{Ayuda \machineName}\par}
    \vfill % Rellena los espacios vacios en los documentos

    % --------------------------------------------------------
    \includegraphics[width=\textwidth,height=8cm,keepaspectratio]{\logoMachine}\par\vspace{1cm}
    \vfill

    % --------------------------------------------------------
        \begin{tcolorbox}[colback=red!5!white,colframe=red!75!black]
            \centering
            mybx\\maquinita maquinita
        \end{tcolorbox}
        \vfill\vfill

    % --------------------------------------------------------
    {\large \startDate\par}
    % \vfill

    % --------------------------------------------------------
    \end{titlepage}

    % --------------------------------------------------------
    % Comienzo del TOC
    \clearpage
    \tableofcontents
    \clearpage
    % --------------------------------------------------------
    % Tabla de figuras
    \clearpage
    \listoffigures
    \clearpage
    % --------------------------------------------------------
   \section{Antecedentes} 
   El presente docummento recoge los resultados obtenidos duratne la fase de auditoria realizada 
   a la maquina {\textbf\machineName} de la plataforma \href{https://caasimilas.com}{\textbf{\color{blue}Caasimilas}}.  

    % --------------------------------------------------------
   \vspace{0.2cm}
   
    % --------------------------------------------------------
    \begin{figure}[h]
    \centering
        \includegraphics[width=\textwidth]{~/null/2021-11-07-121417_1366x768_scrot.png}
        \caption{Detalles de la maquina}
    \end{figure}

    % --------------------------------------------------------
    % --------------------------------------------------------
    \begin{tcolorbox}[enhanced,attach boxed title to top center={yshift=-3mm,yshifttext=-1mm},
      colback=blue!5!white,colframe=blue!75!black,colbacktitle=greenPortada!80!black,
      title=Direccionamiento,fonttitle=\bfseries,
      boxed title style={size=small,colframe=red!50!black} ]

        \centering   
        \href{https://caasimilas.com}{\color{blue}{https://caasimilas.com}}
    \end{tcolorbox}

    % --------------------------------------------------------
    \vspace{0.1cm}

    % --------------------------------------------------------
   \clearpage

    % --------------------------------------------------------
   \section{Objetivos}
    Conocer el estado de seguridad actual del servidor \textbf{\machineName}, enumerando posibles vectores de explotacion y determinando el alcance e impacto que un atacante podria ocasionar sobre el sistema en produccion
    \subsection{Consideraciones}
    Una vez finalizadas las jornadas de auditoria, se llevara a cabo una fase de saneamiento y buenas practicas con el objetivo de securizar el servidor y evitar ser victimas de un futuro ataque en base a los vectores explotados 

    % --------------------------------------------------------
    \vspace{0.5cm}
    % --------------------------------------------------------
    \begin{figure}[h]
       \begin{center}
           \smartdiagram[priority descriptive diagram]{
               Reconocimiento sobre el sistema,
               Deteccion de vulnerabilidades,
               Explotacion de vulnerabilidades,
               Securizacion del sistema
           } 
       \end{center} 
    \caption{flujo de trabajo}
    \end{figure}
    % --------------------------------------------------------
    \clearpage
    % --------------------------------------------------------
    \section{Analisis de vulnerabilidades}
    \subsection{Reconocimiento inicial}
    \vspace{0.2cm}
   Se comenzo realizando un analisis inicial sobre el sistema, verificando que el sistema objetivo se encontrara accesible desde el segmento de red en el que se opera: 
    % --------------------------------------------------------
   \vspace{0.2cm}
   \begin{figure}[h]
      \begin{center}
          \includegraphics[width=0.4\textwidth]{~/null/2021-11-08-155804_256x411_scrot.png}
      \end{center} 
       \caption{Mi master}
   \end{figure}
    % --------------------------------------------------------
   \par
    \vspace{0.2cm}
    % --------------------------------------------------------
    Una vez localizado, se realizo un escaneo a traves de la herramienta \textbf{nmap} para la deteccion de puertos abiertos, obteniendo los siguientes resultados:

    % --------------------------------------------------------
    \begin{figure}[h]
       \begin{center}
           \includegraphics[width=0.5\textwidth]{~/Sin nombre.png}
       \end{center} 
        \caption{Codigo}
    \end{figure}
    % --------------------------------------------------------
    \clearpage
    % --------------------------------------------------------
    Asimismo, con el objetivo de determinar falsos positivos, se disenio un script en bash para enumerar posibles puertos adicionales para que la herramienta nmpa no llegara a detectar:

    % --------------------------------------------------------
    \vspace{0.2cm}
    % --------------------------------------------------------

    \begin{lstlisting}[language=python, caption=Script personalizado para la enumeracion de puertos]
       for i in formato:
            print(i)
    \end{lstlisting}
    % --------------------------------------------------------
    \vspace{0.3cm}
    % --------------------------------------------------------
    A traves de este script, fue posible detectar puertos adicionalmente abiertos
    % --------------------------------------------------------
    \begin{schema}{TCP}
       Puertos
        \where
        593, 1337
    \end{schema}

    % --------------------------------------------------------
    Una vez finalizada la fase de enumeracion de puertos, se detectaron los servicion y versiones que corrian bajo estos,, representando a continuacion los mas significativos bajo los cuales fue posible explotar el sistema:
    % --------------------------------------------------------
    \begin{figure}[h]
       \centering 
        \makebox[\textwidth]{\includegraphics[width=0.9\paperwidth]{~/null/2021-11-08-182248_1366x768_scrot.png}}
        \caption{Enumeracion de servicios y versiones}
        \label{fig:servicesResults} % para poder referenciarla luego
    \end{figure}

    \vspace{0.2cm}

    Tal y como se aprecio en la figura \ref{fig:servicesResults} de la pagina \pageref{fig:servicesResults}, es posible identificar que se trata de una maquina con \textbf{Directorio activo} configurado.  % forma de referenciar las figuras estando lejos de ellas y refenciar la pagina en que se encuentran

\end{document}
